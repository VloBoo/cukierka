% Микро TODO по оформлению, вырезки из "оформления кп и дп". В планах потом все эти настройки вынести в отдельный шаблон и для этого шаблона отдульную репу

% [x] Шрифт Times New Roman
% [ ] размер шрифта – 14 пт 
% [x] с использованием одинарного интервала в форматах

% [x] Абзацный отступ (первая строка) для всего текста пояснительной записки (включая заголовки разделов и подразделов, перечисления) должен быть единым и составлять 1,25 см. 
% [x] Для основного текста пояснительной записки должно быть установлено выравнивание – по ширине. 
% [x] левое поле – 3,0 см; правое поле – 1,5 см; верхнее поле – 2,0 см; нижнее поле – 2,0 см;

% [ ] По тексту не должно быть висячих строк, т.е. когда одна-две строки абзаца (первая или последняя) переходит на отдельную страницу.
% [ ] Также не допускается висячих подзаголовков или пунктов.
% [ ] В конце листа можно оставить не более пяти пустых строк, если далее не следует новый раздел.

% [ ] Проверить переносы

% [ ] титульный лист;
% [ ] содержание;
% [ ] введение;
% [ ] основная часть; 
% [ ] заключение;
% [ ] список использованных источников; 
% [ ] приложение.

% [ ] Содержание включающее номера и наименование разделов и подразделов с указанием номеров листов (страниц).
% [ ] Отточие не ставится.
% [ ] Слово «СОДЕРЖАНИЕ» записывают в виде заголовка (симметрично тексту) прописными буквами.
% [ ] Наименования, включенные в содержание, записывают строчными буквами, начиная с прописной буквы. 

% [ ] Номера страниц не проставляются на титульном листе, листе задания. На листе «СОДЕРЖАНИЕ» номер страницы указывается в основной надписи.
% [ ] Каждый новый раздел записки необходимо начинать с новой страницы
% [ ] Заголовок раздела, подраздела оставлять на листе без текста не допускается. 

% [ ] Разделы должны иметь порядковые номера в пределах всей пояснительной записки, 
% [ ] обозначенные арабскими цифрами без точки и записанные с абзацного отступа, 
% [ ] применяется гарнитура шрифта Times New Roman
% [ ] Каждый раздел, подраздел и пункт записывают с абзацного отступа.
% [ ] Разделы, подразделы должны иметь заголовки.
% [ ] Расстояние между заголовком и основным текстом при выполнении пояснительной записки или отчета должно быть равно 28 пунктов.
% [ ] Расстояние между заголовками раздела и подраздела 14 пунктов.
% [ ] Расстояние между текстом и названием подраздела – 28 пунктов.

% ОП Т.093026
\documentclass[12pt,a4paper,draft]{report}

\usepackage{cmap}
\usepackage[T2A]{fontenc}
\usepackage[utf8]{inputenc}
\usepackage[english,russian]{babel}
\usepackage{fontspec}
\usepackage[left=3cm,right=1.5cm,
    top=2cm,bottom=2cm,bindingoffset=0cm]{geometry}

\setmainfont{Times New Roman}
\setmonofont{Courier New} % Моноширинный шрифт

\linespread{1}
\parindent=1.25cm

\title{Отчёт по технологической практике}
\author{Владислав Черняков}

\begin{document}

\maketitle

\tableofcontents

\chapter{Организационно-функциональная структура предприятия и подразделения, характеристики основных видов деятельности}

ООО "ВЭБ Технологии" представляет собой инновационную информационно-технологическую компанию, специализирующуюся в разработке программного обеспечения для различных предприятий.
Основными клиентами являются компании из различных отраслей, что обуславливает разнообразие выпускаемого софта.

Структура компании организована в соответствии с принципами функциональной дифференциации.
На вершине иерархии находится Генеральный директор, подчиненные ему отделы ответственны за различные аспекты деятельности.

Основные виды деятельности ООО "ВЭБ Технологии" включают в себя:
\begin{enumerate}
    \item Разработка программного обеспечения для корпоративных заказчиков;
    \item Интеграция информационных систем;
    \item Консультационные услуги по внедрению IT-решений;
    \item Техническая поддержка и сопровождение программных продуктов.
\end{enumerate}

Практика осуществляется в отделе разработки, который является ключевым звеном в производственном процессе компании.
Подразделение включает в себя следующие должности:
\begin{enumerate}
    \item Руководитель отдела разработки: Отвечает за общее управление и координацию разработки программного обеспечения.
    \item Программисты (разработчики): Занимаются написанием и тестированием кода для создания программных продуктов.
    \item Тестировщики: Осуществляют проверку функциональности программного обеспечения, выявляют и устраняют ошибки.
    \item Аналитики: Занимаются анализом требований заказчика, выявляют ключевые особенности проекта.
\end{enumerate}

Отдел разработки является частью департамента по инновационным технологиям, который напрямую подчинен исполнительному директору компании.
Департамент сосредотачивает в себе ключевые компетенции в области информационных технологий.

Отдел разработки занимается созданием и совершенствованием программного обеспечения в соответствии с требованиями заказчиков.
Он активно взаимодействует с другими отделами компании, такими как отделы аналитики и тестирования, для обеспечения высокого качества и функциональности разрабатываемых продуктов.

\chapter{Должностные обязанности}

Должостная инструкция инженера-программиста:
\begin{enumerate}
    \item Общие положения
    \begin{enumerate}
        \item Инженер-программист (программист) относится к категории специалистов, принимается на работу и увольняется с работы приказом руководителя организации по представлению ООО "ВЭБ Технологии"
        
        \item На должность инженера-программиста (программиста) назначается лицо, имеющее высшее профессиональное (техническое или инженерно-математическое, математическое) образование без предъявления требований к стажу работы или среднее специальное (техническое или инженерно-математическое, математическое) образование и стаж работы в должности техника-программиста категории не менее 3 лет, либо других должностях, замещаемых специалистами со средним специальным образованием, не менее 5 лет.

        На должность инженера-программиста (программиста) II категории назначается лицо, имеющее высшее профессиональное (техническое или инженерно-математическое, математическое) образование и стаж работы в должности инженера-программиста (программиста) или других инженерно-технических должностях, замещаемых специалистами с высшим профессиональным образованием, не менее 3 лет.
    
        На должность инженера-программиста (программиста) I категории назначается лицо, имеющее высшее профессиональное (техническое или инженерно-математическое, математическое) образование и стаж работы в должности инженера-программиста (программиста) II категории не менее 3 лет.

        \item В своей деятельности инженер-программист (программист) руководствуется:
        \begin{itemize}
            \item нормативными документами по вопросам выполняемой работы;
            \item методическими материалами, касающимися соответствующих вопросов;
            \item уставом организации;
            \item правилами трудового распорядка;
            \item приказами и распоряжениями руководителя организации (непосредственного руководителя);
            \item настоящей должностной инструкцией.
        \end{itemize}
        
        \item Инженер-программист (программист) должен знать:
        \begin{itemize}
            \item руководящие и нормативные материалы, регламентирующие методы разработки алгоритмов, программ и использования вычислительной техники при обработке информации;
            \item основные принципы структурного программирования;
            \item виды программного обеспечения;
            \item технико-эксплуатационные характеристики, конструктивные особенности, назначение и режимы работы ЭВМ, правила их технической эксплуатации;
            \item технологию автоматизированной обработки информации;
            \item виды технических носителей информации;
            \item методы классификации и кодирования информации;
            \item формализованные языки программирования;
            \item действующие стандарты, системы счислений, шифров и кодов;
            \item порядок оформления технической документации;
            \item передовой отечественный и зарубежный опыт программирования и использования вычислительной техники;
            \item основы экономики, организации производства, труда и управления;
            \item основы трудового законодательства;
            \item правила и нормы охраны труда и пожарной безопасности.
        \end{itemize}

        \item Во время отсутствия инженера-программиста (программиста) его обязанности выполняет в установленном порядке назначаемый заместитель, несущий полную ответственность за надлежащее исполнение возложенных на него обязанностей.

    \end{enumerate}

    \item Функции

    На инженера-программиста (программиста) возлагаются следующие функции:
    \begin{enumerate}
        \item Разработка программ, направленных на решение экономических и иных задач.
        \item Осуществление запуска и отладка программ.
        \item Сопровождение внедренных программ и программных средств.
        \item Участие в разработке форм документов, подлежащих машинной обработке.
        \item Освоение и применение в работе новых компьютерных технологий.
    \end{enumerate}

    \item Должостные обязанности

    Для выполнения возложенных на него функций инженер-программист (программист) обязан:
    \begin{enumerate}
        \item На основе анализа математических моделей и алгоритмов решения научных, прикладных экономических и других задач разрабатывать программы, обеспечивающие возможность выполнения алгоритма и соответственно поставленной задачи средствами вычислительной техники, проводить их отладку и тестирование.
        \item Разрабатывать технологию решения задачи по всем этапам обработки информации.
        \item Осуществлять выбор языка программирования для описания алгоритмов и структур данных.
        \item Определять информацию, подлежащую обработке средствами вычислительной техники, ее объемы, структуру, макеты и схемы ввода, обработки, хранения и вывода, методы ее контроля.
        \item Выполнять работу по подготовке программ к отладке и проводить отладку.
        \item Определять объем и содержание данных контрольных примеров, обеспечивающих наиболее полную проверку соответствия программ их функциональному назначению.
        \item Осуществлять запуск отлаженных программ и ввод исходных данных, определяемых условиями поставленных задач.
        \item Проводить тестирование и корректировку разработанной программы на основе анализа выходных данных.
        \item Разрабатывать инструкции по работе с программами, оформлять необходимую техническую документацию.
        \item Осваивать и применять в работе новые компьютерные технологии.
        \item Определять возможность использования готовых программных продуктов.
        \item Осуществлять сопровождение внедренных программ и программных средств.
        \item Разрабатывать и внедрять системы автоматической проверки правильности программ, типовые и стандартные программные средства, составлять технологию обработкиинформации.
        \item Выполнять работу по унификации и типизации вычислительных процессов.
        \item Принимать участие в создании каталогов и картотек стандартных программ, в разработке форм документов, подлежащих машинной обработке, в проектировании программ, позволяющих расширить область применения вычислительной техники.
    \end{enumerate}

    \item Права
    
    Инженер-программист (программист) имеет право:
    \begin{enumerate}
        \item Знакомиться с проектами решений руководства организации, касающимися его деятельности.
        \item Вносить на рассмотрение руководства предложения по совершенствованию работы, связанной с обязанностями, предусмотренными настоящей инструкцией.
        \item Получать от руководителей структурных подразделений, специалистов информацию и документы, необходимые для выполнения своих должностных обязанностей.
        \item Привлекать специалистов всех структурных подразделений организации для решения возложенных на него обязанностей (если это предусмотрено положениями о структурных подразделениях, если нет - с разрешения руководителя организации).
        \item Требовать от руководства организации оказания содействия в исполнении своих должностных обязанностей и прав.
    \end{enumerate}

    \item Взаимоотношения (связи по долности)
    
    Nope.

    \item Оценка работы и ответственность
    
    \begin{enumerate}
        \item Работу инженера-программиста (программиста) оценивает непосредственный руководитель (иное должностное лицо).
        \item Инженер-программист (программист) несет ответственность:
        \begin{enumerate}
            \item За неисполнение (ненадлежащее исполнение) своих должностных обязанностей, предусмотренных настоящей должностной инструкцией, в пределах, определенных действующим трудовым законодательством Республики Беларусь.
            \item За совершенные в процессе осуществления своей деятельности правонарушения - в пределах, определенных действующим административным, уголовным и гражданским законодательством Республики Беларусь.
            \item За причинение материального ущерба - в пределах, определенных действующим трудовым, уголовным и гражданским законодательством Республики Беларусь.
        \end{enumerate}
    \end{enumerate}

\end{enumerate}

\chapter{Программное обеспечение, используемое на предприятии}

На предприятии широко используется разнообразное программное обеспечение, ориентированное на эффективную разработку, тестирование и поддержку программных продуктов.
Важной составляющей инфраструктуры является операционная система, система контроля версий, язык программирования, система управления базой данных, инструменты для тестирования и другие средства, обеспечивающие полный жизненный цикл программного продукта.

Операционная система Windows является основным средством поддержки для работы с компьютерами на предприятии.
Эта операционная система предоставляет удобное и функциональное окружение для разработчиков и других сотрудников, обеспечивая стабильность и совместимость с различными прикладными программами.

Система контроля версий Git используется для эффективного управления исходным кодом проектов.
Она обеспечивает возможность отслеживания изменений, совместной работы разработчиков, а также управления версиями программного обеспечения, что является ключевым элементом современного программирования.

Прикладной язык программирования Java используется для создания масштабируемых и кросс-платформенных приложений.
Этот язык предоставляет широкий спектр инструментов для разработки, обеспечивает высокую производительность и широкую совместимость с различными технологиями.

Система управления базой данных PostgreSQL является выбором для хранения и обработки данных на предприятии.
Она отличается открытым исходным кодом, высокой производительностью и надежностью, что делает ее предпочтительным решением для проектов с различными масштабами и сложностью.

Инструмент Postman используется для тестирования API-интерфейсов, обеспечивая автоматизированное тестирование и проверку функциональности взаимодействия между различными компонентами программного обеспечения.

Система тестирования JUnit предоставляет средства для модульного тестирования кода на языке Java.
Она позволяет создавать и запускать тесты, обеспечивая контроль качества разрабатываемых программных продуктов.

Таким образом, применение указанных программных средств на предприятии обеспечивает надежную и эффективную основу для разработки, тестирования и сопровождения программных продуктов, соответствуя высоким стандартам современной индустрии информационных технологий.

\chapter{Реализация индивидуального задания}

\section{Исследование предметной области}

В контексте месте прохождения практики было решено реализовать программный продукт для предприятия ООО "ВЭБ Технологии" с целью управления проектами в рамках бизнес-среды.
Проектируемое приложение предназначено для эффективного ведения проектов, предполагающих активное взаимодействие сотрудников и систематизацию задач.

Инструментальным средством данного приложения будет веб-сайт, обеспечивающий сотрудникам возможность создания и управления доской задач.
Эта доска, представляющая собой специальную таблицу, позволяет каждому участнику процесса определить и структурировать задачи, которые требуется выполнить.
Дополнительно, рассматривается вариант использования баг-трекера, предназначенного для систематизации и учета выявленных ошибок.

Важным этапом в разработке данного программного продукта является проведение исследования предметной области.
Это включает в себя анализ существующих задач, подлежащих автоматизации, и обоснование необходимости компьютерной обработки информации для повышения эффективности процессов.
Также рассматриваются формы выходных документов, такие как договоры, счета, счет-фактуры и отчеты, предоставляя возможность визуализации структуры выходной информации.

Описывается текущий процесс решения задачи в организации, а также периодичность использования разрабатываемого программного продукта.
В контексте упрощения решения задачи выделяется функциональность доски задач и баг-трекера, предоставляющих наглядные и структурированные средства взаимодействия и мониторинга задач.

Помимо этого, проводится анализ существующих аналогов, выявляя особенности и преимущества разрабатываемого приложения.
Этот этап позволяет лучше понять уникальные черты предлагаемого решения и определить его конкурентоспособность на рынке бизнес-приложений.

\section{Проектирование модели}

На первом этапе проектирования необходимо определить объекты, которые будут взаимодействовать в рамках создаваемого приложения.
Это включает в себя доску задач, баг-трекер, пользователей и другие элементы, составляющие функциональную структуру системы.
Каждый объект будет иметь свои характеристики и роль в процессе управления проектами.

Для более наглядного представления функциональных требований системы, будет построена диаграмма вариантов использования.
Эта диаграмма позволит систематизировать различные сценарии использования приложения, выделяя ключевые варианты взаимодействия между пользователями и системой.
Данный инструментарий дает возможность лучше понять потребности пользователей и обеспечить соответствие приложения их ожиданиям.

Важным шагом в проектировании веб-приложения является разработка первоначальной структуры.
Для этого используется организационный список или ментальная карта, предоставляя обзор основных разделов, функций и переходов между ними.
Эта структура служит основой для дальнейшего развития и реализации веб-интерфейса, обеспечивая логичную и интуитивно понятную навигацию для пользователей.

Таким образом, процесс проектирования модели программного продукта включает в себя детализацию объектов, вариантов использования и первоначальной структуры, обеспечивая основу для последующей разработки и реализации.

\section{Организация данных}

Логическая модель данных включает в себя описание структуры данных и их взаимосвязей на более высоком уровне абстракции.
В контексте управления проектами, объектами логической модели могут быть, например, проекты, задачи, пользователи и т.д.
Отношения между этими объектами представляют собой ключевые аспекты логической организации данных.

Физическая модель данных, в свою очередь, фокусируется на реальной структуре данных, их типах, индексации и отношениях на уровне базы данных.
В контексте среды разработки, физическая модель может определять таблицы базы данных, их поля и связи между таблицами. Важным аспектом является оптимизация физической структуры для эффективного хранения и обработки данных.

Выходные данные, порождаемые системой, будут представлять собой информацию о проектах, статусах задач, пользователях и других аспектах управления проектами.
Формат выходных данных будет зависеть от конкретных потребностей пользователей и возможностей системы. Это может включать в себя отчеты, графики, статистику и другие формы представления информации.

Таким образом, организация данных включает в себя построение логической и физической моделей данных, обеспечивающих эффективное и структурированное хранение информации, а также определение формата выходных данных для обеспечения полноценной информационной обратной связи с пользователями системы.

\section{Концептуальный прототип}

В системе меню концептуального прототипа выделяются ключевые разделы и функциональные блоки, обеспечивая легкий доступ пользователя к основным функциям приложения.
Диалоговые окна представляют собой визуальные компоненты для ввода данных, отображения результатов и взаимодействия с системой.
Элементы управления, такие как кнопки, поля ввода, выпадающие списки и т.д., акцентируют внимание на важных действиях пользователя.

Прототипы форм и диалоговых окон являются визуальным отображением концептуальных идей, предоставляя пользователям представление о том, как будет выглядеть интерфейс и какие возможности будут доступны.
Для веб-приложений и сайтов прототипы включают в себя представление страниц и форм, определение расположения элементов, цветовой гаммы и структуры интерфейса.

Процесс разработки концептуального прототипа направлен на создание интуитивно понятного, удобного и привлекательного визуального интерфейса, способного эффективно взаимодействовать с пользователем.
Каждый элемент прототипа тщательно обдумывается с учетом пользовательского опыта, удовлетворяя требованиям функциональности и эстетическим предпочтениям.

Таким образом, концептуальный прототип представляет собой важный этап в разработке программного продукта, обеспечивая предварительное представление о внешнем пользовательском интерфейсе и его функциональности.

\section{Реализация функций}

% Совсем вода уж, надо еще и функции вставить для пример и ссылку на приложение. Переделать потом

Программные модули, представленные в тексте основных модулей, содержат комментарии, поясняющие их функциональность и указывающие на элементы управления, которые инициируют их выполнение.
Это обеспечивает легкость восприятия кода разработчиками и обеспечивает понимание взаимосвязей между интерфейсом и функциональностью программы.

Реализация функций включает в себя взаимодействие с базой данных для эффективного хранения и извлечения данных, обновление информации на основе действий пользователя, а также визуализацию результатов в соответствии с задачами, решаемыми приложением.

Таким образом, в данном разделе представлены реализованные функции, которые непосредственно связаны с элементами управления, обеспечивая работу приложения и взаимодействие пользователя с системой управления проектами.
Комментарии в тексте основных модулей содействуют легкости восприятия и дальнейшей поддержке разработанного программного продукта.

\section{Функциональное тестирование}

Проведение тестирования сопровождается распечатками копий экранов, на которых отражены результаты выполнения определенных действий.
Эти экраны служат визуальной демонстрацией работы программного продукта на различных этапах тестирования.
Результаты работы программного средства, включая экранные формы, отчеты и другие выходные документы, предоставлены в приложении Б.

Тест-кейсы охватывают различные сценарии использования программы, включая основные функции управления проектами, взаимодействие с базой данных, а также корректное отображение информации на пользовательском интерфейсе.
В процессе тестирования проверяется не только функциональность, но и устойчивость системы к возможным ошибкам и сценариям непредвиденного использования.

Таким образом, функциональное тестирование поддерживается тщательно разработанными тест-кейсами, визуальными материалами и результатами выполнения, обеспечивающими адекватную оценку работоспособности программного продукта.

\chapter*{Список использованных источников}

Данная секция пропущена.
В будущем она будет дополнена.

\chapter*{Приложение А Текст программных модулей}

Данная секция пропущена.
В будущем она будет дополнена.

\chapter*{Приложение Б Результаты работы приложения}

Данная секция пропущена.
В будущем она будет дополнена.

\end{document}